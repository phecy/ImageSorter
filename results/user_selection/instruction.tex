\documentclass[10pt,a4paper]{article}
\usepackage[latin1]{inputenc}
\usepackage{amsmath}
\usepackage{amsfonts}
\usepackage{amssymb}
\begin{document}
\title{Photograph Selection}
\author{Armin Samii, Ula Popov}
\maketitle
We are conducting a user study to train and rate the quality of our Automatic Photograph Selection tool. If you are willing to help, please follow these steps:

\begin{enumerate}
\item Find a set of pictures you have taken recently, according to the following parameters:
	\begin{itemize}
	\item \textbf{Consecutive}: In the order they were taken on the camera
	\item \textbf{Complete}: No images were deleted from your computer. Images deleted in-camera are okay, but not preferred.
	\item \textbf{Untouched}: As imported with no retouching.
	\item \textbf{Imperfect}: Don't look for a high quality set, just the most recent that fits the above criteria.
	\item Should be around \textbf{30-50 images} total.
	\end{itemize}
\item Select a \textbf{high-quality, representative subset} from these. Ask yourself, "if I could pick some of these to Photoshop then show my friends/family/client, which would I choose?"
	\begin{itemize}
	\item \textbf{High qualit}y: Well-exposed, not blurry, etc.
	\item \textbf{Representative}: Captures the idea of the set with less images
	\item Can be \textbf{however many images} you find appropriate. Thirty pictures of the beach might only have two pictures in the subset; thirty pictures of a day at a carnival may have fifteen.
	\end{itemize}
\item Email (1) the zipped images and (2) the list of filenames in the subset to \textit{ksamii@ucsc.edu}. We will not share your pictures.
\end{enumerate}

Thank you!

\end{document}