\documentclass[twocolumn]{article}

\title{
   Assessing Image Quality on Import From Camera
} % Work in Progress
\author{
   K. Armin Samii\\
   Computer Science Undergraduate\\
   Univ. of Calif. at Santa Cruz\\
   ksamii@ucsc.edu
  \and
   Allison Carlisle\\
   Biomolecular Engineering Undergraduate\\
   Univ. of Calif. at Santa Cruz\\
   acarlisle@ucsc.edu
  \and
   Uliana Popov\\
   Computer Science Graduate\\
   Univ. of Calif. at Santa Cruz\\
   uliana@soe.ucsc.edu
  \and
   James Davis\\
   Computer Science Professor\\
   Univ. of Calif. at Santa Cruz\\
   davis@soe.ucsc.edu
}
\date{\today}

\begin{document}

\maketitle

\begin{abstract}
In this paper, we propose a novel method for ranking images as they are imported from a users camera. We provide the user with the best image from every scene photographed and a ranking of these scenes. Our goal is to filter images which are technically flawed to allow the photographer to focus on the best image from each similar set. By assuming a chronological import, we rank images relative to each other, eliminating the need for no-reference image based techniques. The set chosen by our algorithm matches the top two choices of photographers in our study with astonishing accuracy (97\%).
\end{abstract}

\section{Introduction}
In general, photographers who wish to polish their photos on a computer need to sort through the images taken during a photoshoot and select the ones most suited for retouching. This process involves first throwing out all technically flawed images, then subjectively choosing their favorites from the remaining set.

Our research aims to replace the computational work required by the human in the first step and allow a computer to perform the tedious work. The second, subjective step is one of personal preference and not the goal of this work.

Various approaches have been proposed to rank a set of random images based on aesthetic quality. Yeh \textit{et al.}\cite{Yeh:2010:PPR:1873951.1873963} ranks any input on highly customizable, subjective ratings based on a user's personal preference, but is only intended for amateur photographers. 

\bibliographystyle{plain}
\bibliography{README_BIB}
\end{document}
